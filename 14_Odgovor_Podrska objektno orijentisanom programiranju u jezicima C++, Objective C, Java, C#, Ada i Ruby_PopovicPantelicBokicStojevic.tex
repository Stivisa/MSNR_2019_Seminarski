

 % !TEX encoding = UTF-8 Unicode

\documentclass[a4paper]{report}

\usepackage[utf8x,utf8]{inputenc} % make weird characters work

\usepackage{amssymb}

\usepackage{color}
\usepackage{url}
\usepackage[unicode]{hyperref}
\hypersetup{colorlinks,citecolor=green,filecolor=green,linkcolor=blue,urlcolor=blue}

\newcommand{\odgovor}[1]{\textcolor{blue}{#1}}

\begin{document}

\title{ Podrška objektno orijentisanom programiranju u jezicima C++, Objective C, Java, C\#, Ada i Ruby\\ \small{Katarina Popović, Dušan Pantelić, Dejan Bokić, Nikola Stojević}}

\maketitle

\tableofcontents

\chapter{Uputstva}
\emph{Prilikom predavanja odgovora na recenziju, obrišite ovo poglavlje.}

Neophodno je odgovoriti na sve zamerke koje su navedene u okviru recenzija. Svaki odgovor pišete u okviru okruženja \verb"\odgovor", \odgovor{kako bi vaši odgovori bili lakše uočljivi.} 
\begin{enumerate}

\item Odgovor treba da sadrži na koji način ste izmenili rad da bi adresirali problem koji je recenzent naveo. Na primer, to može biti neka dodata rečenica ili dodat pasus. Ukoliko je u pitanju kraći tekst onda ga možete navesti direktno u ovom dokumentu, ukoliko je u pitanju duži tekst, onda navedete samo na kojoj strani i gde tačno se taj novi tekst nalazi. Ukoliko je izmenjeno ime nekog poglavlja, navedite na koji način je izmenjeno, i slično, u zavisnosti od izmena koje ste napravili. 

\item Ukoliko ništa niste izmenili povodom neke zamerke, detaljno obrazložite zašto zahtev recenzenta nije uvažen.

\item Ukoliko ste napravili i neke izmene koje recenzenti nisu tražili, njih navedite u poslednjem poglavlju tj u poglavlju Dodatne izmene.
\end{enumerate}

Za svakog recenzenta dodajte ocenu od 1 do 5 koja označava koliko vam je recenzija bila korisna, odnosno koliko vam je pomogla da unapredite rad. Ocena 1 označava da vam recenzija nije bila korisna, ocena 5 označava da vam je recenzija bila veoma korisna. 

NAPOMENA: Recenzije ce biti ocenjene nezavisno od vaših ocena. Na osnovu recenzije ja znam da li je ona korisna ili ne, pa na taj način vama idu negativni poeni ukoliko kažete da je korisno nešto što nije korisno. Vašim kolegama šteti da kažete da im je recenzija korisna jer će misliti da su je dobro uradili, iako to zapravo nisu. Isto važi i na drugu stranu, tj nemojte reći da nije korisno ono što jeste korisno. Prema tome, trudite se da budete objektivni. 
\chapter{Recenzent \odgovor{--- ocena: 4} }


\section{O čemu rad govori?}
% Напишете један кратак пасус у којим ћете својим речима препричати суштину рада (и тиме показати да сте рад пажљиво прочитали и разумели). Обим од 200 до 400 карактера.
U ovom radu su obja\v snjeni osnovni koncepti objektno orijentisanih jezika i njihova implementacija u jezicima C++, Objective C, Java, C\#, Ada i Ruby.
Za svaki jezik posebno su obja\v snjeni enkapsulacija, nasle\dj{}ivanje, polimorfizam i apstrakcija. Navedene su i neke osobine koje dati jezik odvajaju od ostalih.
%Za svaki od ovih jezika je 
\section{Krupne primedbe i sugestije}
% Напишете своја запажања и конструктивне идеје шта у раду недостаје и шта би требало да се промени-измени-дода-одузме да би рад био квалитетнији.
U poglavlju 2, koje se bavi programskim jezikom C++, i poglavlju 5, koje je vezano za C\#, nedostaju reference. 

Trebalo bi razmisliti da li je u delovima 2.3, 3.3, 4.3,  umesto termina premo\v s\'cavanje i koncept va\v znosti metoda, bolje koristiti termin preklapanje (eng. overriding). Tako\dj{}e, mislim da je u poglavlju 5.1 bolje koristiti termine metoda i atribut  umesto funkcija i promenljiva jer je re\v c o pravima pristupa unutar klase.

\odgovor{
U delovima 3.3, 4.3 koje govore o polimorfizmu izmenjen je termin važnosti metoda u termin preklapanja, jer se čini intuitivnijim i adekvatnijim. U poglavlju 5.1 zamenjeni termini predlozenim terminima.}

\section{Sitne primedbe}
% Напишете своја запажања на тему штампарских-стилских-језичких грешки
U Uvodu, kod obja\v snjenja nasle\dj{}ivanja prva re\v cenica nije najjasnija, bilo bi lepo da se preformuli\v se.

Deo 2.2 - "+"dok - nedostaje zatvoren navodnik, umesto slova d stoji slovo \dj{}.

Poglavlje 3 - Svaka klasa je izvedena iz superklase NSObect, \v ciji konstruktor init je podrazumevani konstruktor, mo\v ze predefinisati. - ova re\v cenica je nejasna, potrebno ju je preformulisati. Nije po\v stovano pravilo za razmake posle ta\v cke, zareza i zagrada.

\odgovor{
Preformulisana je problematična rečenica, podeljena je u dve zapravo. Popravljeni razmaci.
''Svaka klasa nasleđuje baznu klasu NSObject, koja sadrži konstruktor init. Ovaj konstruktor se može predefinisati.''
}

Deo 3.3 - Koncept va\v znosti metoda(eng.overriding) postoji, i prikazan je u primeru(3). Gde klase zaposleni i voza\v c imaju isti metod display i poziv [empldisplay] \'ce izvr\v siti metod klase voza\v c, zato \v sto promenljiva zaposleni empl referi\v se na objekat tipa voza\v c. - ovo je potrebno popraviti (predlog, spojiti re\v cenice u jednu ili izbaciti re\v c 'Gde').

\odgovor{
Ranije je već navedena zamerka oko termina važnosti metoda i to je ispravljeno. Nije lepo bilo objašnjeno, popravljene su rečenice u poglavlju 3.3. Izbačena je reč `Gde', ostale su dve rečenice. ''Koncept preklapanja metoda (eng.~\textbf{\em overriding})  postoji, i prikazan je u primeru (3). Klase zaposleni i vozač imaju isti metod display i poziv [empl display] će izvršiti metod klase vozač, zato što promenljiva zaposleni empl referiše na objekat tipa vozač. ''
}

Deo 3.4 - Prvu re\v cenicu bi bilo lepo podeliti u dve. U poslednjoj re\v cenici dodati zareze iza re\v ci @required, klasi i iza re\v ci @optional.

\odgovor{
Podeljena je prva rečenica.
''Jezik Objective C nema definisan koncept apstraktnih klasa [4]. Sličan efekat je moguće postići programerskom snalažljivošću i ne instancirati klasu koja bi trebalo biti apstraktna.''
Ispravljene gramatičke greške sa zarezom.
'' Sekcija protokola se završava sa \textbf{@end} i može sadžati dve podoblasti \textbf{@required}, za metode koji se obavezno moraju implementirati u klasi i \textbf{@optional}, za metode čija je implementacija opciona.''
}

Poglavlje 4 - U prvoj re\v cenici vi\v sak je drugo 'su'.
\odgovor{
Uklonjen je višak. ''Kako su klase u centru zbivanja krenućemo od njih.''
}

Deo 4.4 - Smatram da je u duhu srpskog jezika bolje napisati 'prema \v clanku' nego 'prema artiklu'.
\odgovor{
Usvojena primedba, promenjeno u `prema članku`.  ''Izdvajanje skupa metoda sa kojima spoljašnji korisnik komunicira, prema članku [3], vršimo pomoću apstraktnih klasa ili interfejsa.''
}

Poglavlje 5 - Veznik 'i' pisati malim slovom (gre\v ska u celom poglavlju).
Re\v cenica - Veome je sli\v cna podr\v ska OOP-u kao kod Java programskog jezika takođe su iste I klase I strukture. - nejasno, popraviti (predlog, zarez iza tako\dj{}e).

\odgovor{
Prepravljen veznik 'i' sa velikom na malo slovo u celom poglvalju. Navedena recenica je preformulisana.
}

Deo 5.1 - U prvoj re\v cenici je 'od' vi\v sak. O\v si\v sana latinica na nekoliko mesta (izlozi, clan).
\odgovor{
Uklonjen visak 'od'. Prepravljena latinica.
}

Deo 5.2 - O\v si\v sana latinica (moze, sto).
\odgovor{
Prepravljena latinica.
}


Deo 5.3 - U prvoj re\v cenici jedno 'da' je vi\v sak.
\odgovor{
Uklonjen visak 'da'.
}

Deo 5.4 - O\v si\v sana latinica (nasih).
\odgovor{
Prepravljena latinica.
}

Poglavlje 6 - Nisu po\v stovana pravila o razmacima pre i posle zagrade (gre\v ska u celom poglavlju).

Poglavlje 7 - Nisu po\v stovana pravila o razmacima pre i posle zagrade (gre\v ska u celom poglavlju).

Deo 7.1 - O\v si\v sana latinica (mogucnost). Potreban je zarez posle public.

Deo 7.2 - O\v si\v sana latinica (nasledjuje).

Deo 7.3 - O\v si\v sana latinica (sto, ce, razliciti)

Deo 7.4 - O\v si\v sana latinica (pokrecu, pomocu, olaksava). Potreban je razmak posle "NotImplementedError".

Poglavlje 8 - U poslednjoj re\v cenici izostavljeno slovo 'h' (programski(h) jezika).

U delovima 4.4 i 5.3 postoje pasusi koji sadr\v ze samo jednu re\v cenicu.
\odgovor{
U poglavlju 4.4 izdvojena je rečenica u pasus je predstavlja opšti uvod. Možda nije efektivno, pa je promenjeno. 
}

\section{Provera sadržajnosti i forme seminarskog rada}
% Oдговорите на следећа питања --- уз сваки одговор дати и образложење

\begin{enumerate}
\item Da li rad dobro odgovara na zadatu temu?\\
Da. Pokriveni su osnovni koncepti objektno orijentisanih jezika, obja\v snjeni su na\v cini za njihovo ostvarivanje kao i razlike u odnosu na druge objektno orijentisane jezike.
\item Da li je nešto važno propušteno?\\
Nije.
\item Da li ima suštinskih grešaka i propusta?\\
Ima manjih gre\v saka i propusta koji su gore navedeni
\item Da li je naslov rada dobro izabran?\\
Naslov rada je isti kao naslov teme.
\item Da li sažetak sadrži prave podatke o radu?\\
Sadr\v zi. Na osnovu sa\v zetka je jasno \v cime se rad bavi.
\item Da li je rad lak-težak za čitanje?\\
Rad je lak za \v citanje.
\item Da li je za razumevanje teksta potrebno predznanje i u kolikoj meri?\\
Nije potrebno veliko predznanje. Dovoljno je poznavati osnove oop-a (\v sta su klase, metodi, atributi).
\item Da li je u radu navedena odgovarajuća literatura?\\
Nedostaje literatura za C++ i C\#.
\odgovor{
Dodata literatura za C\#.
}

\item Da li su u radu reference korektno navedene?\\
Da.
\item Da li je struktura rada adekvatna?\\
Jeste.
\item Da li rad sadrži sve elemente propisane uslovom seminarskog rada (slike, tabele, broj strana...)?\\
Sadr\v zi odgovaraju\'ci broj slika, tabela i strana.
\item Da li su slike i tabele funkcionalne i adekvatne?\\
Slika i tabela su funkcionalne i adekvatne za deo teksta u kom se nalaze.
\end{enumerate}

\section{Ocenite sebe}
% Napišite koliko ste upućeni u oblast koju recenzirate: 
% a) ekspert u datoj oblasti
% b) veoma upućeni u oblast
% c) srednje upućeni
% d) malo upućeni 
% e) skoro neupućeni
% f) potpuno neupućeni
% Obrazložite svoju odluku

Srednje upu\'cen.

Poznajem navedene koncepte oop-a u meri u kojoj smo ih izu\v cavali na ranijim kursevima, i imam iskustva sa programiranjem u pojedinim jezicima iz teksta, ali ne u svim.


\chapter{Dodatne izmene}
%Ovde navedite ukoliko ima izmena koje ste uradili a koje vam recenzenti nisu tražili. 

\end{document}
