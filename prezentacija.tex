\documentclass[14pt,aspectratio=169]{beamer}
\usetheme{Marburg}
\graphicspath{{Arquivos/}}
\usepackage[utf8]{inputenc}
\usepackage[portuguese]{babel}
\usepackage[T1]{fontenc}
\usepackage{amsmath}
\usepackage{amsfonts}
\usepackage{amssymb}
\usepackage{graphicx}
\usepackage{bibentry}

\definecolor{mygreen}{rgb}{0,0.6,0}
\definecolor{mygray}{rgb}{0.5,0.5,0.5}
\definecolor{mymauve}{rgb}{0.58,0,0.82}
\usepackage{listings}
\lstset{ 
  backgroundcolor=\color{white},   % choose the background color; you must add \usepackage{color} or \usepackage{xcolor}; should come as last argument
  basicstyle=\scriptsize\ttfamily,        % the size of the fonts that are used for the code
  breakatwhitespace=false,         % sets if automatic breaks should only happen at whitespace
  breaklines=true,                 % sets automatic line breaking
  captionpos=b,                    % sets the caption-position to bottom
  commentstyle=\color{mygreen},    % comment style
  deletekeywords={...},            % if you want to delete keywords from the given language
  escapeinside={\%*}{*)},          % if you want to add LaTeX within your code
  extendedchars=true,              % lets you use non-ASCII characters; for 8-bits encodings only, does not work with UTF-8
  firstnumber=1,                % start line enumeration with line 1000
  frame=single,	                   % adds a frame around the code
  keepspaces=true,                 % keeps spaces in text, useful for keeping indentation of code (possibly needs columns=flexible)
  keywordstyle=\color{blue},       % keyword style
  language=Python,                 % the language of the code
  morekeywords={*,...},            % if you want to add more keywords to the set
  numbers=left,                    % where to put the line-numbers; possible values are (none, left, right)
  numbersep=5pt,                   % how far the line-numbers are from the code
  numberstyle=\tiny\color{mygray}, % the style that is used for the line-numbers
  rulecolor=\color{black},         % if not set, the frame-color may be changed on line-breaks within not-black text (e.g. comments (green here))
  showspaces=false,                % show spaces everywhere adding particular underscores; it overrides 'showstringspaces'
  showstringspaces=false,          % underline spaces within strings only
  showtabs=false,                  % show tabs within strings adding particular underscores
  stepnumber=2,                    % the step between two line-numbers. If it's 1, each line will be numbered
  stringstyle=\color{mymauve},     % string literal style
  tabsize=2,	                   % sets default tabsize to 2 spaces
  title=\lstname                   % show the filename of files included with \lstinputlisting; also try caption instead of title
}
\author{Katarina Popović, Dušan Pantelić, Dejan Bokić, Nikola Stojević}
\newcommand{\TT}{Podrška objektno orijentisanom programiranju u jezicima C++, Objective C, Java, C\#, Ada i Ruby}
\newcommand{\TB}{Texto Bíblico}
\newcommand{\DT}{Deuteronômio 26.1-15}

\title{Podrška objektno orijentisanom programiranju u jezicima C++, Objective C, Java, C\#, Ada i Ruby}
%\setbeamercovered{transparent} 
\institute{\small{Seminarski rad u okviru kursa\\Metodologija stručnog i naučnog rada\\ Matematički fakultet}}
\date{Maj 2019} 
%\subject{} 
\begin{document}

\begin{frame}
\titlepage
\end{frame}

\begin{frame}{Sadržaj}
	\tableofcontents
\end{frame}

\section{Uvod}
\begin{frame}[fragile]{Uvod}
\begin{itemize}
\item Programska paradigma
\item Princip jedinstvene odgovornosti
\item Enkapsulacija
\item Nasleđivanje
\item Polimorfizam
\item Apstrakcija
\end{itemize}
\end{frame}

\section{C++}
\begin{frame}[fragile]{C++}
\begin{itemize}
\item C++ je delimično objektno orijentisan jezik
	\begin{itemize}
		\item Main funkcija izvan klase
		\item Koncept globalne promenljive
		\item Postojanje friend funkcija
	\end{itemize}
\item Enkapsulacija u C++
	\begin{itemize}
		\item public, protected i private sekcije
	\end{itemize}
\item Nasleđivanje u C++
	\begin{itemize}
		\item public, protected i private nasleđivanje
		\item virtuelno nasleđivanje
	\end{itemize}
\item Polimorfizam u C++
	\begin{itemize}
		\item polimorfizam u vreme kompilacije
		\item polimorfizam u vreme izvršavanja
	\end{itemize}
\item Apstrakcija u C++
\end{itemize}
\end{frame}


\section{Objective C}
\begin{frame}[fragile]{Objective C}
\begin{lstlisting}[caption={Primer koda u Objective C jeziku},frame=single, label=ObjectiveC]
@interface Employee : NSObject {
   double salary;	@public int age;}
@property(nonatomic, readwrite) double salary; 
- (void)display;
@end # `-' za metode instance, `+' za klasne metode(static)
@implementation Employee
@synthesize salary; 
- (void)display { NSLog(@"Employee salary is %f", salary); }
@end # (id) tip koji je kompaktibilan svakom objektu
@interface Driver : Employee { NSString* truck; }
- (id)initWithTruck:(NSString*)model;
@end # self oznacava tekuci objekat
@implementation Driver
- (id)initWithTruck:(NSString*)model {
   truck = model;	return self;  }
- (void)display { NSLog(@"Driver salary is %f", salary); } @end
int main(int argc, const char * argv[]) {
   NSAutoreleasePool * pool = [[NSAutoreleasePool alloc] init];
   Employee *empl = [[Driver alloc]initWithTruck:@"Mercedes"];
   empl.salary = 5.0;  empl->age = 33;
   [empl display];
   [pool drain];
   return 0;
}}
\end{lstlisting}
\end{frame}

\section{Java}
\begin{frame}[fragile]{Java \small{- primer koda sa enkapsulacijom, nasledjivanjem, polimorfizmom} }
\begin{lstlisting}
public class Employee {
	private int salary;
	#this je referenca na tekuci objekat
	public Employee(int salary) { this.salary = salary;}
 	public int getSalary(){ return salary;}
	public void setSalary(int newSalary) { salary = newSalary;}
	public void display() {
     		System.out.println("Hello i'm employee!");
   }
	public static void main(String[] args) {
    		Employee Marko = new Driver(600,"Mercedes");
    		Marko.display();}    
}
class Driver extends Employee {
  	String truck = "FAP";
	#super vrsi poziv konstruktora bazne klase
   	public Driver(int salary,String truck) {
		super(salary); this.truck = truck;}
  	public void display() {
		System.out.println("My truck is "+truck+"!");
	public void display(String x) {
		System.out.println("My truck is "+truck+x+"!");
}
\end{lstlisting}
\end{frame}

\begin{frame}[fragile]{Apstrakcija}
Apstraktne klase ili interfjesi
\begin{lstlisting}
public abstract class Employee {
	public abstract void display(); ...
interface Employee {
	public void display(); #podrazumevano apstraktna
	default void work(){System.out.println("Working"); }
\end{lstlisting}
\begin{table}
\begin{center}
\caption{Vidljivost različitih modifikatora pristupa.}
\begin{tabular}{|l|c|c|c|c|} \hline
Modifikator &Klasa &Paket &Podklasa &Svet\\ \hline
public &Da &Da &Da &Da\\ \hline
protected &Da &Da &Da &Ne\\ \hline
podrazumevani &Da &Da &Ne &Ne\\ \hline
private &Da &Ne &Ne &Ne\\ \hline
\end{tabular}
\label{tab:tabelaModPristupa}
\end{center}
\end{table}
\end{frame}


\section{Literatura}
\begin{frame}[fragile]{Literatura}
\begin{itemize}
	\scriptsize
	\item Introduction to Ada. on-line at: \url{https://learn.adacore.com/courses/intro-to-ada/index.html}
	\item Object C apple documentation. on-line at: \url{https://developer.apple.com/library/archive/documentation/Cocoa/Conceptual/ObjectiveC}
	\item Ruby - Object Oriented. on-line at: \url{https://www.tutorialspoint.com/ruby/ruby_object_oriented.htm}
	\item Gary Bennet, Brad Lees and MItchell Fisher. Objective-C for Absolute Beginners: IPhone, iPad and Mac Programming Made Easy. Apress, Berkely, CA, USA, 3rd edition, 2016
	\item AdaCore experts. High-Integrity Object-Oriented Programming in Ada. AdaCore(www.adacore.com), 1.2 release edition, 2011. on-line at: \url{http://extranet.eu.adacore.com/articles/HighIntegrityAda.pdf}
	\item Hal Fulton. The Ruby Way. Sams Publishing, 2001.
	\item Cay S Horstmann. Core Java SE 9 for the Impatient. Addison-Wesley Professional, 2017.
	\item Aayushi Johari. Object Oriented Programming - Java OOPs Concepts With Examples, 2018. on-line at: \url{https://www.edureka.co/blog/object-oriented-programming/}
	\item Stephen Prata. C++ Primer Plus (5th Edition) (Primer Plus (Sams)). Sams, Indianapolis, IN, USA, 2004
\end{itemize}
\bibliographystyle{unsrt}
\end{frame}


\end{document}