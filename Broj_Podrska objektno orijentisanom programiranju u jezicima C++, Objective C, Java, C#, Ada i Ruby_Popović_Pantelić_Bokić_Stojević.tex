% !TEX encoding = UTF-8 Unicode
\documentclass[a4paper]{article}

\usepackage{color}
\usepackage{url}
\usepackage[T2A]{fontenc} % enable Cyrillic fonts
\usepackage[utf8]{inputenc} % make weird characters work
\usepackage{graphicx}

\usepackage[english,serbian]{babel}
%\usepackage[english,serbianc]{babel} %ukljuciti babel sa ovim opcijama, umesto gornjim, ukoliko se koristi cirilica

\usepackage[unicode]{hyperref}
\hypersetup{colorlinks,citecolor=green,filecolor=green,linkcolor=blue,urlcolor=blue}

\usepackage{listings}

%\newtheorem{primer}{Пример}[section] %ćirilični primer
\newtheorem{primer}{Primer}[section]

\definecolor{mygreen}{rgb}{0,0.6,0}
\definecolor{mygray}{rgb}{0.5,0.5,0.5}
\definecolor{mymauve}{rgb}{0.58,0,0.82}

\lstset{ 
  backgroundcolor=\color{white},   % choose the background color; you must add \usepackage{color} or \usepackage{xcolor}; should come as last argument
  basicstyle=\scriptsize\ttfamily,        % the size of the fonts that are used for the code
  breakatwhitespace=false,         % sets if automatic breaks should only happen at whitespace
  breaklines=true,                 % sets automatic line breaking
  captionpos=b,                    % sets the caption-position to bottom
  commentstyle=\color{mygreen},    % comment style
  deletekeywords={...},            % if you want to delete keywords from the given language
  escapeinside={\%*}{*)},          % if you want to add LaTeX within your code
  extendedchars=true,              % lets you use non-ASCII characters; for 8-bits encodings only, does not work with UTF-8
  firstnumber=1000,                % start line enumeration with line 1000
  frame=single,	                   % adds a frame around the code
  keepspaces=true,                 % keeps spaces in text, useful for keeping indentation of code (possibly needs columns=flexible)
  keywordstyle=\color{blue},       % keyword style
  language=Python,                 % the language of the code
  morekeywords={*,...},            % if you want to add more keywords to the set
  numbers=left,                    % where to put the line-numbers; possible values are (none, left, right)
  numbersep=5pt,                   % how far the line-numbers are from the code
  numberstyle=\tiny\color{mygray}, % the style that is used for the line-numbers
  rulecolor=\color{black},         % if not set, the frame-color may be changed on line-breaks within not-black text (e.g. comments (green here))
  showspaces=false,                % show spaces everywhere adding particular underscores; it overrides 'showstringspaces'
  showstringspaces=false,          % underline spaces within strings only
  showtabs=false,                  % show tabs within strings adding particular underscores
  stepnumber=2,                    % the step between two line-numbers. If it's 1, each line will be numbered
  stringstyle=\color{mymauve},     % string literal style
  tabsize=2,	                   % sets default tabsize to 2 spaces
  title=\lstname                   % show the filename of files included with \lstinputlisting; also try caption instead of title
}

\begin{document}

\title{\Large Podrška objektno orijentisanom programiranju u jezicima C++, Objective C, Java, C\#, Ada i Ruby\\ \small{Seminarski rad u okviru kursa\\Metodologija stručnog i naučnog rada\\ Matematički fakultet}}

\author{Katarina Popović, Dušan Pantelić, Dejan Bokić, Nikola Stojević\\ kontakt email prvog, pantelic.dusan@protonmail.com, trećeg, četvrtog autora}

%\date{9.~april 2015.}

\maketitle

\abstract{
U ovom tekstu je ukratko prikazana osnovna forma seminarskog rada. Obratite pažnju da je pored ove .pdf datoteke, u prilogu i odgovarajuća .tex datoteka, kao i .bib datoteka korišćena za generisanje literature. Na prvoj strani seminarskog rada su naslov, apstrakt i sadržaj, i to sve mora da stane na prvu stranu! Kako bi Vaš seminarski zadovoljio standarde i očekivanja, koristite uputstva i materijale sa predavanja na temu pisanja seminarskih radova. Ovo je samo šablon koji se odnosi na fizički izgled seminarskog rada (šablon koji \emph{morate} da koristite!) kao i par tehničkih pomoćnih uputstava. Pročitajte tekst pažljivo jer on sadrži i važne informacije vezane za zahteve obima i karakteristika seminarskog rada.}

\tableofcontents

\newpage

\section{Uvod}
\label{sec:uvod}

Kada budete predavali seminarski rad, imenujete datoteke tako da sadrže redni broj teme, temu seminarskog rada, kao i prezimena članova grupe. Precizna uputstva na temu imenovnja će biti data na formi za predaju seminarskog rada. Predaja seminarskih radova biće isključivo preko veb forme, a NE slanjem mejla. Link na formu će biti dat u okviru obaveštenja na strani kursa. Vodite računa da prilikom predavanja seminarskog rada predate samo one fajlove koji su neophodni za ponovno generisanje pdf datoteke. To znači da pomoćne fajlove, kao što su .log, .out, .blg, .toc, .aux i slično, \textbf{ne treba predavati}.

\newpage

\section{C++}
\label{sec:c++}
Deo za C++.

\section{Objective C}
\label{sec:objectivec}
Deo za Objective C.

\section{Java}
\label{sec:java}

Objekte klasa instanciramo pomoću metoda konstruktora(nema povratni tip i uvek se zove isto kao i klasa) sa odgovarajućim argumentima. Ako ne definišemo konstruktor, automatski se generiše podrazumevani konstruktor, koji je prazan i nema argumente. U slučaju da nema argumente, inicijalizuje objekat na podrazumevane vrednosti. Sledeći primer (\ref{lst:javaDeklaracija}) predstavlja deklaraciju klase zaposleni, koja ima svoje atribute i metode(detaljnije \ref{subsec:javaEnkapsulacija}).

\begin{lstlisting}[caption={Primer deklarisanja klase sa enkapsulacijom i nasleđivanjem},frame=single, label=lst:javaDeklaracija]
public class Employee {
	private int salary;
	#this je referenca na tekuci objekat
	public Employee(int salary) { this.salary = salary;}
 	public int getSalary(){ return salary;}
	public void setSalary(int newSalary) { salary = newSalary;}
	public void display() {
     		System.out.println("Hello i'm employee!");
   }
	public static void main(String[] args) {
    		Employee Marko = new Driver(600,"Mercedes");
    		Marko.display();}    
}
class Driver extends Employee {
  	String truck = "FAP";
	#super vrsi poziv konstruktora bazne klase
   	public Driver(int salary,String truck) {
		super(salary); this.truck = truck;}
  	public void display() {
		System.out.println("My truck is "+truck+"!");
	public void display(String x) {
		System.out.println("My truck is "+truck+x+"!");
}
\end{lstlisting}
\subsection{Enkapsulacija}
\label{subsec:javaEnkapsulacija}

Ograničavanje pristupa internim podacima klase postižemo navođenjem ključne reči ~{\em \textbf{private}} ispred deklaracije promenljive u klasi. Ovo znači da se podacima može pristupiti isključivo iz deklarisane klase. Podacima neophodnim za funkcionalnost programa obezbeđuje se pristup čitanja i menjanja (eng. ~{\em getters and setters})\cite{horstmann2017core} preko javnih metoda. U primeru koda (\ref{lst:javaDeklaracija}), vrednosti privatnog atributa plata možemo pristupiti metodom getSalary() ili menjati sa setSalary(newSalary).

\subsection{Nasleđivanje}
\label{subsec:javaNasledjivanje}

Za označavanje koristimo ključnu reč \textbf{extends}. Podela po artiklu \cite{oopJava}:
\begin{itemize}
  \item Po nivoima, kada klasu A nasleđuje klasa B, a nju nasleđuje klasa C.
  \item Hijerarhijsko nasleđivanje, gde klase B i C nasleđuju klasu A.
  \item Višestruko nasleđivanje(nasleđivanje više klasa) nije moguće, već se implementira preko interfejsa(detaljnije \ref{subsec:javaApstrakcija}).
\end{itemize}
U primeru koda (\ref{lst:javaDeklaracija}), klasa vozač nasleđuje klasu zaposleni.

\subsection{Polimorfizam}
\label{subsec:javaPolimorfizam}

Višestruka upotrebljivost koda za različite vrste objekata.

Pripadnost metoda objektu  se obavlja u vreme izvršavanja(eng.~{\em run time execution}) i predstavlja koncept važnosti metoda(eng.~{\em\textbf{overriding}})\cite{horstmann2017core}. U primeru koda \ref{lst:javaDeklaracija}, Marko.display(); pozvaće metod klase vozač. 

Koncept prenatrpanosti metoda(eng.~{\em\textbf{overloading}})\cite{horstmann2017core}, određuje metode u vremenu kompajliranja(eng.{\em compile time}) na osnovu razlika u potpisu metode(različito ime metoda ili tipovi i broj parametara). U primeru koda [\ref{lst:javaDeklaracija}], Marko.display(2); pozvaće metod display(int x) klase vozač.

\subsection{Apstrakcija}
\label{subsec:javaApstrakcija}

Izdvajanje skupa metoda sa kojima spoljašnji korisnik komunicira, prema artiklu \cite{oopJava}, vršimo pomoću apstraktnih klasa ili interfejsa.

Za apstraktne klase navodimo ključnu reč~{\em \textbf{abstract}}(kod \ref{lst:javaAbstract}). Ne mogu se instancirati, ali može biti tip promenljive. Sadrže apstraktne metode(istom ključnom reči obeležavaju) koje treba da predefiniše neka podklasa.

\begin{lstlisting}[caption={Apstraktna klasa},frame=single, label=lst:javaAbstract]
public abstract class Employee {
	public abstract void display(); ...
\end{lstlisting}

Interfejs predstavlja nacrt klase. Sadrži apstraktne, statične, podrazumevane metode(mogu se predefinisati u klasi) i statičke promenljive. Da implementiramo interfejs navodimo ključnu reč~{\em \textbf{implements}}(kod \ref{lst:javaInterfejs}) i zatim ime interfejsa(slično nasleđivanju). Prednost interfejsa\cite{horstmann2017core} je da klasa može implementirati više interfejsa, dok može da nasleđuje samo jednu klasu.

\begin{lstlisting}[caption={Interfejs},frame=single, label=lst:javaInterfejs]
interface Employee {
	public void display(); #podrazumevano apstraktna
	default void work(){System.out.println("Working"); }
class Driver implements Employee{  
   	 public void display(){...}
\end{lstlisting}

\newpage

\section{C\#}
\label{sec:csharp}
Deo za C\#.

\section{Ada}
\label{sec:ada}
Deo za Ada.

\newpage

\section{Ruby}
\label{sec:ruby}
Osnovu objektno orijentisanog programiranja u programskom jeziku Ruby prikazaćemo primerom(\ref{lst:rubyDeklaracija}) kreiranja klase i kreiranjem instanci klase tj. objekata. Standardni metod klase je \textbf{initialize}, on se poziva automatski prilikom kreiranja objekta i ponaša se skoro identično kao konstruktori u drugim programskim jezicima.

\begin{lstlisting}[caption={Primer deklarisanja klase u programskom jeziku Ruby.},frame=single, label=lst:rubyDeklaracija]
class Student
	@@total_num_of_students = 0
	def initialize(name)
		@name = name
		increase_num_of_students()
	end
	private
	def increase_num_of_students()
		@@total_num_of_students += 1
	end
end

stud1 = Student.new("John")


\end{lstlisting}

\subsection{Enkapsulacija}
\label{subsec:rubyEnkapsulacija}
Kako u samom jeziku ne postoji mogućnost direktnog pristupa podacima unutar klase(podaci su privatni), njima možemo pristupiti jedino pomoću metoda klase. Svi metodi klase su javni, osim ako nije eksplicitno naznačeno drugačije. Postoje tri nivoa zaštite pristupa: javni(eng. ~{\em public}), zasticeni(eng. ~{\em protected}) i privatni(eng. ~{\em private}). Navođenjem jednog od tri prethodno navedena nivoa u liniji neposredno pre definicije metoda postižemo željeni efekat.

\subsection{Nasleđivanje}
\label{subsec:rubyNasledjivanje}

\subsection{Polimorfizam}
\label{subsec:rubyPolimorfizam}

\subsection{Apstrakcija}
\label{subsec:rubyApstrakcija}

\section{Osnovna uputstva}
Vaš seminarski rad mora da sadrži najmanje jednu \textbf{sliku}, najmanje jednu \textbf{tabelu} i najmanje \textbf{sedam referenci} u spisku literature. Najmanje jedna slika treba da bude originalna i da predstavlja neke podatke koje ste Vi osmislili da treba da prezentujete u svom radu. Isto važi i za najmanje jednu tabelu. 	Od referenci, neophodno je imati bar jednu \textbf{knjigu}, bar jedan \textbf{naučni članak} iz odgovarajućeg časopisa i bar jednu adekvatnu \textbf{veb adresu}. 

\textbf{Dužina seminarskog rada treba da bude od 10 do 12 strana.} Svako prekoračenje ili potkoračenje biće kažnjeno sa odgovarajućim brojem poena. Eventualno, nakon strane 12, može se javiti samo tekst poglavlja \textbf{Dodatak} koji sadrži nekakav dodatni k\^{o}d, ali je svakako potrebno da rad može da se pročita i razume i bez čitanja tog dodatka. 

Ко жели, може да пише рад ћирилицом. У том случају, неопходно је да су инсталирани одговарајући пакети: texlive-fonts-extra, texlive-latex-extra, texlive-lang-cyrillic, texlive-lang-other. 

Nemojte koristiti stari način pisanja slova, tj ovo:
\begin{verbatim}
\v{s} i \v{c} i \'c ...
\end{verbatim}
Koristite direknto naša slova:	
\begin{verbatim}
š i č i ć ... 
\end{verbatim}


\section{Engleski termini i citiranje}	
\label{sec:termini_i_citiranje}

Na svakom mestu u tekstu naglasiti odakle tačno potiču informacije. Uz sve novouvedene termine u zagradi naglasiti od koje engleske reči termin potiče. 

Naredni primeri ilustruju način uvođenja enlegskih termina kao i citiranje. \cite{horstmann2017core}

\begin{primer}
Problem zaustavljanja (eng.~{\em halting problem}) je neodlučiv \cite{haltingproblem}.
\end{primer}

\begin{primer}
Za prevođenje programa napisanih u programskom jeziku C može se koristiti GCC kompajler \cite{gcc}.
\end{primer}

\begin{primer}
 Da bi se ispitivala ispravost softvera, najpre je potrebno precizno definisati njegovo ponašanje \cite{laski2009software}. 
\end{primer}

Reference koje se koriste u ovom tekstu zadate su u datoteci {\em seminarski.bib}. Prevođenje u pdf format u Linux okruženju može se uraditi na sledeći način:
\begin{verbatim}
pdflatex TemaImePrezime.tex 
bibtex TemaImePrezime.aux 
pdflatex TemaImePrezime.tex 
pdflatex TemaImePrezime.tex 
\end{verbatim}
Prvo latexovanje je neophodno da bi se generisao {\em .aux} fajl. {\em bibtex} proizvodi odgovarajući {\em .bbl} fajl koji se koristi za generisanje literature. 
Potrebna su dva prolaza (dva puta pdflatex) da bi se reference ubacile u tekst (tj da ne bi ostali znakovi pitanja umesto referenci). Dodavanjem novih referenci potrebno je ponoviti ceo postupak.  











Broj naslova i podnaslova je proizvoljan. Neophodni su samo Uvod i Zaključak. Na poglavlja unutar teksta referisati se po potrebi. 
\begin{primer}
U odeljku \ref{sec:naslov1} precizirani su osnovni pojmovi, dok su zaključci dati u odeljku \ref{sec:zakljucak}.
\end{primer}

Još jednom da napomenem da nema razloga da pišete:
\begin{verbatim}
\v{s} i \v{c} i \'c ...
\end{verbatim}
Možete koristiti srpska slova
\begin{verbatim}
š i č i ć ... 
\end{verbatim}



\section{Slike i tabele}
\label{slike_i_tabele}

Slike i tabele treba da budu u svom okruženju, sa odgovarajućim naslovima, obeležene labelom da koje omogućava referenciranje. 

\begin{primer} Ovako se ubacuje slika. Obratiti pažnju da je dodato i 
\begin{verbatim}
\usepackage{graphicx}
\end{verbatim}

Na svaku sliku neophodno je referisati se negde u tekstu. Na primer, na slici \ref{fig:pande} prikazane su pande. 
\end{primer}

\begin{primer} I tabele treba da budu u svom okruženju, i na njih je neophodno referisati se u tekstu. Na primer, u tabeli \ref{tab:tabela1} su prikazana različita poravnanja u tabelama.

\begin{table}[h!]
\begin{center}
\caption{Razlčita poravnanja u okviru iste tabele ne treba koristiti jer su nepregledna.}
\begin{tabular}{|c|l|r|} \hline
centralno poravnanje& levo poravnanje& desno poravnanje\\ \hline
a &b&c\\ \hline
d &e&f\\ \hline
\end{tabular}
\label{tab:tabela1}
\end{center}
\end{table}

\end{primer}

\section{K\^{o}d i paket listings}
Za ubacivanje koda koristite paket \textbf{listings}:
\url{https://en.wikibooks.org/wiki/LaTeX/Source_Code_Listings}

\begin{primer}
Primer ubacivanja koda za programski jezik Python dat je kroz listing \ref{simple}. Za neki drugi programski jezik, treba podesiti odgvarajući programski jezik u okviru defnisanja stila.
\end{primer}
\begin{lstlisting}[caption={Primer ubacivanja koda u tekst},frame=single, label=simple]
# This program adds up integers in the command line
import sys
try:
    total = sum(int(arg) for arg in sys.argv[1:])
    print 'sum =', total
except ValueError:
    print 'Please supply integer arguments'
\end{lstlisting}


\section{Prvi naslov}
\label{sec:naslov1}


Ovde pišem tekst. 
Ovde pišem tekst. 
Ovde pišem tekst. 
Ovde pišem tekst. 
Ovde pišem tekst. 
Ovde pišem tekst. 
Ovde pišem tekst. 
Ovde pišem tekst. 


\subsection{Prvi podnaslov}
\label{subsec:podnaslov1}

Ovde pišem tekst. 
Ovde pišem tekst. 
Ovde pišem tekst. 
Ovde pišem tekst. 
Ovde pišem tekst. 
Ovde pišem tekst. 
Ovde pišem tekst. 

\subsection{Drugi podnaslov}
\label{subsec:podnaslov2}

Ovde pišem tekst. 
Ovde pišem tekst. 
Ovde pišem tekst. 
Ovde pišem tekst. 
Ovde pišem tekst. 
Ovde pišem tekst. 


\subsection{... podnaslov}
\label{subsec:podnaslovN}

Ovde pišem tekst. 
Ovde pišem tekst. 
Ovde pišem tekst. 
Ovde pišem tekst. 
Ovde pišem tekst. 
Ovde pišem tekst. 

\section{n-ti naslov}
\label{sec:naslovN}

Ovde pišem tekst. 
Ovde pišem tekst. 
Ovde pišem tekst. 
Ovde pišem tekst. 
Ovde pišem tekst. 

\subsection{... podnaslov}
\label{subsec:podnaslovK}

Ovde pišem tekst. 
Ovde pišem tekst. 
Ovde pišem tekst. 
Ovde pišem tekst. 
Ovde pišem tekst. 

\subsection{... podnaslov}
\label{subsec:podnaslovM}

Ovde pišem tekst. 
Ovde pišem tekst. 
Ovde pišem tekst. 
Ovde pišem tekst. 
Ovde pišem tekst. 


\section{Zaključak}
\label{sec:zakljucak}

Ovde pišem zaključak. 
Ovde pišem zaključak. 
Ovde pišem zaključak. 
Ovde pišem zaključak. 
Ovde pišem zaključak. 
Ovde pišem zaključak. 
Ovde pišem zaključak. 
Ovde pišem zaključak. 
Ovde pišem zaključak. 
Ovde pišem zaključak. 
Ovde pišem zaključak. 
Ovde pišem zaključak. 


\addcontentsline{toc}{section}{Literatura}
\appendix
\bibliography{seminarski} 
\bibliographystyle{plain}

\appendix
\section{Dodatak}
Ovde pišem dodatne stvari, ukoliko za time ima potrebe.
Ovde pišem dodatne stvari, ukoliko za time ima potrebe.
Ovde pišem dodatne stvari, ukoliko za time ima potrebe.
Ovde pišem dodatne stvari, ukoliko za time ima potrebe.
Ovde pišem dodatne stvari, ukoliko za time ima potrebe.


\end{document}
